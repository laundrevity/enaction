\documentclass{article}


% Required packages
\usepackage{amsmath}   % For mathematical symbols
\usepackage{amssymb}   % For additional symbols
\usepackage{amsthm}    % For theorem-like environments
\usepackage{tikz-cd}   % For commutative diagrams
\usepackage{hyperref}  % For hyperlinks
\usepackage{cleveref}  % For smart cross-referencing
\usepackage{geometry}  % For page layout
\usepackage{enumitem}  % For custom lists
\usepackage{graphicx}  % For including images
\usepackage{mathtools} % For additional mathematical tools
\usepackage{hyperref}  % For links
\usepackage{mdframed}

\newtheorem{definition}{Definition}[section]
\newtheorem{remark}{Remark}[section]
\newtheorem{theorem}{Theorem}[section]
\newtheorem{example}{Example}[section]
\newtheorem{lemma}{Lemma}[section]
\newtheorem{proposition}{Proposition}[section]
\newtheorem{corollary}{Corollary}[section]

% Define a custom quote environment with a vertical bar on the left
\newmdenv[
    leftline=true, 
    topline=false, 
    bottomline=false, 
    rightline=false, 
    linecolor=gray, 
    linewidth=3pt,
    skipabove=\topsep,
    skipbelow=\topsep
]{customquote}

\title{Adjunctive cognition: A topos-theoretic framework for enactivism}
\author{Conor L. Mahany\footnote{
    Email: conor.mahany@proton.me. Barring some unforeseen circumstances, the most recent version of this work can be found here \href{https://github.com/laundrevity/enaction/blob/master/enaction.pdf}{https://github.com/laundrevity/enaction/blob/master/enaction.pdf}.
}}
\date{\today}

\begin{document}

% Title
\maketitle

% Abstract
\begin{abstract}
    This work presents a novel framework for understanding cognition through the lens of category theory and topos theory. Moving beyond traditional cognitivist models that emphasize internal representations of an external world, cognition is reimagined here as a dynamic adjunction between action and perception. By conceptualizing the self as the initial object in a category of contexts, parallels are drawn to the Buddhist concept of \emph{śūnyatā} (emptiness), offering a bridge between Western mathematical structures and Eastern philosophical insights. Concrete examples illustrate the framework, and its implications for cognitive science, philosophy, and future research are explored.
\end{abstract}

\newpage
% Quotes section

\begin{customquote}
    If I were to say that the so-called philosophy of this fellow Hegel is a colossal piece of mystification which will yet provide posterity with an inexhaustible theme for laughter at our times, that it is a pseudo-philosophy paralyzing all mental powers, stifling all real thinking, and, by the most outrageous misuse of language, putting in its place the \textbf{hollowest}, most senseless, thoughtless, and, as is confirmed by its success, most stupefying verbiage, I should be quite right.” \\
    \begin{center}
        Arthur Schopenhauer, \emph{On the Basis of Morality} (1839)
    \end{center}
\end{customquote}

\vspace{10px} % Add some vertical space between quotes

\begin{customquote}
    The slogan is ``\textbf{Adjoint functors arise everywhere}''. \\
    \begin{center}
        Saunders Mac Lane, \emph{Categories for the Working Mathematician} (1971)
    \end{center}
\end{customquote}

\vspace{10px}

\begin{customquote}
    ... the experience of persons interacting is always an inter-experience, and so, we must consider your experience of the relationship. Your frown may, indeed, indicate hostility, but your hostility may be a function of your perceiving me as having betrayed your confidence, which, however, I may not experience myself as having done. Further, I may not realise that you see me as having done that, while you may or may not know that I do not see you as seeing me in this way. \textbf{Unless our experience of each other is clarified a spiral of reciprocal fear, mistrust and misunderstanding will build up.} \\
    \begin{center}
        Aaron Esterson, \emph{The Affirmation of Experience: A contribution towards a science of social situations} (1985) \cite{esterson1985}
    \end{center}
\end{customquote}

\vspace{10px}


\begin{customquote}
    But just as the Madhyamaka dialectic, a provisional and conventional activity of the relative world, points beyond itself, so we might hope that \textbf{our concept of enaction could}, at least for some cognitive scientists and perhaps even for the more general milieu of scientific thought, \textbf{point beyond itself to a truer understanding of groundlessness}.

    \begin{center}
        Varela, Thompson, \& Rosch, \emph{The Embodied Mind} (1991) \cite{varela1991}
    \end{center}
\end{customquote}

\vspace{10px} % Add some vertical space between quotes

\begin{customquote}
    \textbf{Form is emptiness, emptiness is form.}\\

    \begin{center}
        \emph{Heart Sūtra} of the Mahayana Buddhist traditions
    \end{center}
\end{customquote}

\newpage




% Table of contents
\tableofcontents

% Section 1
\section{Introduction}

Understanding cognition remains one of the most challenging and profound quests in both science and philosophy. Traditional Western approaches often model cognition as the manipulation of internal representations of an external world, emphasizing a computational view of the mind. In contrast, enactivism, an emerging paradigm in cognitive science, posits that cognition arises through a dynamic interaction between an organism and its environment, without reliance on internal representations \cite{varela1991}.

In this work, I develop a rigorous mathematical framework for enactivism using tools from category theory and topos theory. These mathematical structures illuminate the dynamic interplay between action and perception, capturing the essence of enaction as an adjunction. Morever, recognizing the self as the initial object within the category of contexts reveals a natural alignment with the Buddhist concept of \emph{śūnyatā} or emptiness \cite{garfield1995}.

The goals in this work are threefold:
\begin{enumerate}
    \item To introduce and justify the use of category theory and topos theory in modeling cognitive processes.
    \item To offer concrete examples that demonstrate how action and perception can be modeled as functors between categories.
    \item To explore the philosophical implications of this framework, particularly the connection between the empty self and \emph{śūnyatā}, and its potential impact on cognitive science.

\end{enumerate}

\subsection{Structure of the work}
\begin{enumerate}[start=2,label=\S]
    \item 2. Provide philosophical motivation, discussing limitations of representational models and introducing enactivism.
    \item 3. Introduce necessary mathematical architecture in category theory and topos theory.
    \item 4. Express cognition as an adjunction between action and perception.
    \item 5. Revisit self as initial object and explore connection to \emph{śūnyatā}.
    \item 6. Discuss implications for cognitive science and philosophy.
    \item 7. Relate to existing literature, highlighting similarities and differences.
    \item 8. Outline future research directions, including empirical validation and theoretical refinement.
    \item 9. Conclude by summarizing key findings and significance.
\end{enumerate}

\section{Philosophical motivation}
Western philosophical thought, particularly in the works of Locke, Descartes, and later Frege and Russell, has traditionally conceived of cognition as a representational process. The mind is often viewed as a container of representations of the external world, processed in logical or computational ways.

However, modern developments, from Wittgenstein’s linguistic turn to Cohen’s forcing and Kripke’s modal semantics, have shifted the conversation. In cognitive science, this shift mirrors a growing emphasis on relational models of cognition, such as enactivism. This work attempts to reconcile the extremes of rigid, representational models and the excessive flexibility of entirely subjective models by proposing a middle way, similar to the balance found in Eastern philosophical traditions.

\subsection{Limitations of representational models}
Building upopn this philosophical backdrop, it's essential to scrutinize the foundational assumptions of representational models in cognitive science. Traditional cognitive science often relies on the assumption that cognition involves internal representations of an external world. This view, rooted in computationalism, treats the mind as a symbol-manipulating system \cite{fodor1980}. However, this model faces several challenges:

\begin{itemize}
    \item \textbf{Frame Problem:} Determining which aspects of a complex and ever-changing environment are relevant to a given context is computationally intractable. This leads to inefficiencies and an inability to adapt swiftly to novel situations \cite{pylyshyn1987}.
    \item \textbf{Embodiment:} Representational models often neglect the profound influence of the body and its interactions with the environment in shaping cognitive processes. Embodied cognition research demonstrates that bodily states are integral to mental functions, a dimension inadequately captured by purely representational systems \cite{clark1997}.
    \item \textbf{Dynamic Interaction:} Cognition is not a passive reception of representations but involves continuous and reciprocal interactions between an organism and its environment. Representational models, with their emphasis on static internal states, fail to account for this dynamic co-evolution \cite{thompson2007}.
\end{itemize}

These inherent limitations of representational models pave the way for enactivism, which emphasizes the inseparable interplay between an organism and its environment. Moreover, these critiques resonate with Eastern philosophical perspectives that challenge inherent notions of representation and existence.

\subsection{Enactivism}

Enactivism offers an alternative by proposing that cognition arises through an organism's active engagement with its environment. Key principles include:

\begin{itemize} \item Autonomy: Cognitive systems are self-organizing and self-maintaining. \item Embodiment: The body shapes the mind, and cognition cannot be separated from the physical form. \item Embeddedness: Cognition is situated within and cannot be isolated from the environment. \item Dynamic Co-emergence: Organism and environment co-determine each other in a continuous feedback loop \cite{varela1991}. \end{itemize}

\subsection{Category theory in enactivism}
The use of category theory, particularly topos theory, offers a rigorous and formal structure to express enactivism's core principles. Enaction, at its heart, emphasizes the dynamic interplay between action and perception, a process that can be expressed as an adjunction between functors. In category theory, functors preserve structure between categories, and adjunctions reflect a reciprocal relationship. In the same way, cognition emerges --- or, dependently arises --- as a reciprocal relationship between an agent's actions and perceptions.

Topos theory, which generalizes set theory and logic, allows us to express contexts and behaviors as categories with their own internal logics. This brings clarity to the relational and context-dependent nature of cognition, paralleling the way enactivism rejects a static, representational self. This bridge provides a novel interdisciplinary method to formalize cognition, without reducing it to either extreme of objectivism or subjectivism.

\subsection{The middle way between rigidity and flexibility}
Traditional approaches in Western philosophy have struggled to reconcile the dichotomy between rigidity and flexibility. Wittgenstein's \emph{Tractatus} represents an approach mired in excessive rigidity, where the world is treated as a system of facts laid out in logical space, and meaning is strictly determined by reference. In contrast, \emph{Philosophical Investigations} swings toward excessive flexibility, where meaning is entirely a matter of use, creating the potential for a kind of interpretative nihilism.

The ``middle way'' proposed here avoids these extremes, much like certain paths in Eastern philosophy that navigate between objectivism and nihilism. This approach aligns well with the Madhyamaka tradition in Buddhism --- which holds that all phenomena are empty of inherent existence --- but this does not lead to nihilism. Instead, it leads to a relational understanding of phenomena. This work seeks to balance the formal structure of topos theory with the flexibility of enactivism by understanding cognition as the adjunction between action and perception.

\subsection{\emph{Śūnyatā} and initial object}
The \emph{initial object} $I$ in a category $\mathcal{C}$ is the object $I$\footnote{If it exists.} with the universal property that there is a unique arrow $I \to X$. The universality condition means that for any object $J$ with a unique arrow $J \to X$ there is a unique arrow $k \colon I \to J$ such that
\[
    \begin{tikzcd}
        I \arrow[r, "k"] \arrow[d] & J \arrow[ld] \\
        X
    \end{tikzcd}
\]
commutes; $J \to X$ ``factors through'' $I \to X$ ``via'' $k$. The \emph{terminal object} is obtained by flipping the directions of all of the arrows.

In the category of contexts, if the arrow structure is causal then the initial object $0$ is the context that is prior to all other contexts, that is, the unique context from which any other context is possible: \textbf{the self}. And the terminal object is the final context, which one might call God or the universe.

In the category of sets \textbf{Set}, the initial object is the empty set $\emptyset$, as for any set $A$ we have a unique arrow $\emptyset \to A$, namely, the empty arrow $\emptyset \subset \emptyset \times A$. The singleton $\{ * \}$ is the terminal object, as there is only one function $A \to \{ * \}$ for any set $A$, namely, the function $a \mapsto *$.

\subsection{Wigner's puzzle and Maslow's hammer}
Eugene Wigner famously puzzled over the ``unreasonable effectiveness'' of mathematics in the natural sciences, pointing out that mathematical concepts seem to fit the physical world in surprisingly fruitful and precise ways, even in contexts where this figt appears coincidental. Wigner likened this to having a bunch of keys and always finding the right one for every door, despite not understanding why the key fits.

Maslow's hammer -- the notion that ``if all you have is a hammer, then everything looks like a nail'' --- warns against overapplying a single tool or framework to every problem. The author's past work \cite{mahany2012} on topos theory admittedly biases them towards believing that it can be used ``fruitfully'' scientifically. One \emph{always} does well to recall that cognitive sophistication does not attenuate the bias blind spot \cite{west2012}, and in fact it likely increases it.

But on the other hand, if all you have is a hammer, then why bother with anything that \emph{isn't} a nail? Nails are things to hammer: Wigner's puzzle is actually no puzzle at all. Lived experience and the ``natural world'' are different aspects of the same fundamental process, so one should expect to find similarity in their structure. That similarity is described by the enactment of cognition (e.g., as an adjunction).

\subsection{Preview}

Topos theory offers a generalization of set theory and its foundations. The fundamental relation becomes not the structural, hierarchical relation $\in$ of set membership but the arrow $\to$ between objects. This is the heart of how to go from a fixed, representational self in to a relational understanding of identity.

Let \textbf{Con} denote the category of contexts and let \textbf{Beh} denote the category of behaviors. Then cognition can be expressed as an adjunction $\mathcal{A} \dashv \mathcal{P}$, where

\begin{itemize}
    \item action is expressed as a functor $\mathcal{A} \colon \textbf{Con} \to \textbf{Beh}$
    \item perception is expressed as a function $\mathcal{P} \colon \textbf{Beh} \to \textbf{Con}$.
\end{itemize}

That is, cognition $\Psi$ consists of a bijection
\[
    \Psi_{CB} \colon \text{Hom} ( \mathcal{A}(C), B ) \cong \text{Hom}(C, \mathcal{P}(B))
\]
that is natural\footnote{Make this precise or don't say it.} in both $C$ and $B$, for any context $C$ and behavior $B$.

For each transformation of action such that one responds to $C$ not with $\mathcal{A}(C)$ but with $B$, there is a unique transformation of perception such that one perceives $B$ not as $\mathcal{P}(B)$ but as $C$. More plainly, ``be the change you want to see in the world''.

In what follows, I will demonstrate that the diagram
\[
    \begin{tikzcd}
        & \{ x \mid \varphi(x) \}  \arrow[r] \arrow[d, tail] & 1 \arrow[d, tail, "\text{true}"] \\
        U \arrow[r, "\alpha"] \arrow[ru, dashed, "m"] & X \arrow[r, "\varphi(x)"] & \Omega
    \end{tikzcd}
\]
is an \emph{operationalizable} characterization of local truth: $U$ \emph{forces} $\varphi(x)$, or $U \vDash \phi(\alpha)$.

\section{Mathematics}
In this section, we introduce the mathematical concepts necessary for our framework, aiming to make them accessible through intuitive explanations and examples.

\subsection{Category Theory}

\begin{definition}
    A \textbf{category} $\mathcal{C}$ consists of:
    \begin{enumerate}[label=(\alph*)]
        \item A collection of \textbf{objects} $X, Y, Z, \dots$.
        \item For each pair of objects $X$ and $Y$, a collection of \textbf{morphisms} (or \textbf{arrows}) $f \colon X \to Y$.
        \item For each object $X$, an \textbf{identity morphism} $1_X \colon X \to X$.
        \item A \textbf{composition law} that assigns to each pair of morphisms $f \colon X \to Y$ and $g \colon Y \to Z$ a morphism $g \circ f \colon X \to Z$.
    \end{enumerate}
    These must satisfy the following axioms:
    \begin{enumerate}
        \item \textbf{Associativity}: For morphisms $f \colon W \to X$, $g \colon X \to Y$, and $h \colon Y \to Z$, we have $h \circ (g \circ f) = (h \circ g) \circ f$.
        \item \textbf{Identity}: For every morphism $f \colon X \to Y$, $1_Y \circ f = f$ and $f \circ 1_X = f$.
    \end{enumerate}
\end{definition}

A prototypical example of a category is $\mathbf{Set}$, the category whose objects are sets and whose morphisms are functions between sets.

\subsubsection{Categories of interest}
\begin{definition}
    Let $\mathbf{Con}$ denote the \textbf{category of contexts}, where:
    \begin{enumerate}
        \item \textbf{Objects}: Each object is a moment of conscious experience.
        \item \textbf{Morphisms}: Arrows represent causal structures or transitions between contexts.
    \end{enumerate}
\end{definition}

\begin{definition}
    Let $\mathbf{Beh}$ denote the \textbf{category of behaviors}, where:
    \begin{enumerate}
        \item \textbf{Objects}: Each object is a behavior in lived experience.
        \item \textbf{Morphisms}: Arrows represent causal structures or transitions between behaviors.
    \end{enumerate}
\end{definition}


\subsubsection{Functors}

\begin{definition} A \textbf{functor} $F \colon \mathcal{C} \to \mathcal{D}$ between categories $\mathcal{C}$ and $\mathcal{D}$ assigns:

    \begin{itemize} \item To each object $X$ in $\mathcal{C}$, an object $F(X)$ in $\mathcal{D}$. \item To each morphism $f \colon X \to Y$ in $\mathcal{C}$, a morphism $F(f) \colon F(X) \to F(Y)$ in $\mathcal{D}$. \end{itemize}

    Such that:

    \begin{enumerate} \item $F(\text{id}_X) = \text{id}_{F(X)}$. \item $F(g \circ f) = F(g) \circ F(f)$. \end{enumerate} \end{definition}

Functors are structure preserving arrows between categories in the sense that they commute with composition.

\subsection{Limits and Colimits}

Understanding limits and colimits is essential for grasping the structural aspects of categories that model cognitive processes.

\begin{definition}
    A \textbf{limit} of a diagram $D \colon \mathcal{J} \to \mathcal{C}$ in a category $\mathcal{C}$ is a universal cone to $D$. Concretely, it consists of an object $L$ in $\mathcal{C}$ and a family of morphisms $\{\pi_j \colon L \to D(j)\}_{j \in \mathcal{J}}$ such that for every morphism $f \colon j \to k$ in $\mathcal{J}$, $D(f) \circ \pi_j = \pi_k$. Moreover, for any other cone $(N, \{\phi_j \colon N \to D(j)\})$, there exists a unique morphism $u \colon N \to L$ such that $\pi_j \circ u = \phi_j$ for all $j \in \mathcal{J}$.
\end{definition}

\begin{definition}
    A \textbf{colimit} is the dual notion of a limit. It is a universal cocone from a diagram $D \colon \mathcal{J} \to \mathcal{C}$.
\end{definition}

\subsubsection{Finite Limits}

\begin{definition}
    A category $\mathcal{C}$ has \textbf{finite limits} if it has all limits of finite diagrams. This includes the existence of products, equalizers, and a terminal object.
\end{definition}

\begin{definition}
    A \textbf{pullback} (or fibered product) of two morphisms $f \colon X \to Z$ and $g \colon Y \to Z$ in a category $\mathcal{C}$ is a limit of the diagram consisting of $X$, $Y$, and $Z$ with $f$ and $g$ as the morphisms to $Z$. It is denoted by:
    \[
        \begin{tikzcd}
            P \arrow[r] \arrow[d] & Y \arrow[d, "g"] \\
            X \arrow[r, "f"] & Z
        \end{tikzcd}
    \]
    where $P$ is the pullback object.
\end{definition}

In $\mathbf{Set}$, the pullback of $f$ and $g$ is the set $\{ (x,y) \in X \times Y \mid f(x) = g(y) \}$.

\subsection{Monomorphisms and Epimorphisms}

\begin{definition}
    A morphism $f \colon X \to Y$ in a category $\mathcal{C}$ is a \textbf{monomorphism} (or \textbf{mono}) if for all objects $Z$ and all pairs of morphisms $g_1, g_2 \colon Z \to X$, $f \circ g_1 = f \circ g_2$ implies $g_1 = g_2$.
\end{definition}

\begin{definition}
    A morphism $f \colon X \to Y$ in a category $\mathcal{C}$ is an \textbf{epimorphism} (or \textbf{epi}) if for all objects $Z$ and all pairs of morphisms $g_1, g_2 \colon Y \to Z$, $g_1 \circ f = g_2 \circ f$ implies $g_1 = g_2$.
\end{definition}

In $\mathbf{Set}$, monomorphisms are injective functions, and epimorphisms are surjective functions.

\subsection{Equalizers and Coequalizers}

\begin{definition}
    Given two parallel morphisms $f, g \colon X \rightrightarrows Y$ in a category $\mathcal{C}$, an \textbf{equalizer} of $f$ and $g$ is a morphism $e \colon E \to X$ such that $f \circ e = g \circ e$, and for any morphism $m \colon M \to X$ with $f \circ m = g \circ m$, there exists a unique morphism $u \colon M \to E$ such that $e \circ u = m$.
\end{definition}

\begin{definition}
    A \textbf{coequalizer} of two parallel morphisms $f, g \colon X \rightrightarrows Y$ is a morphism $q \colon Y \to Q$ such that $q \circ f = q \circ g$, and for any morphism $m \colon Y \to M$ with $m \circ f = m \circ g$, there exists a unique morphism $u \colon Q \to M$ such that $u \circ q = m$.
\end{definition}

\subsection{Subobjects and Subobject Classifier}

\begin{definition}
    A \textbf{subobject} of an object $X$ in a category $\mathcal{C}$ is an equivalence class of monomorphisms into $X$, where two monomorphisms $f \colon A \to X$ and $g \colon B \to X$ are considered equivalent if there exists an isomorphism $h \colon A \to B$ such that $f = g \circ h$.
\end{definition}

\begin{definition}
    A \textbf{subobject classifier} in a category $\mathcal{C}$ is an object $\Omega$ together with a monic $\text{true} \colon 1 \to \Omega$ such that for every subobject of $X$, represented by some monic $m \colon A \to X$, there exists a unique morphism $\chi_m \colon X \to \Omega$ making the following diagram a pullback:
    \[
        \begin{tikzcd}
            A \arrow[r] \arrow[d, tail, "m"] & 1 \arrow[d, tail, "\text{true}"] \\
            X \arrow[r, "\chi_m"] & \Omega
        \end{tikzcd}
    \]
\end{definition}

In $\mathbf{Set}$, the subobject classifier is the two-element set $1 + 1 = \{\text{true}, \text{false}\}$, and $\chi_m$ maps each element of $X$ to $\text{true}$ if it is in the subset $A$ and to $\text{false}$ otherwise.

\subsection{Topos Theory}

\begin{definition}
    A \textbf{topos} $\mathcal{E}$ is a category that satisfies the following properties:
    \begin{enumerate}
        \item $\mathcal{E}$ has all finite limits.
        \item $\mathcal{E}$ has exponentials; that is, for any objects $X$ and $Y$ in $\mathcal{E}$, there exists an exponential object $Y^X$.
        \item $\mathcal{E}$ has a subobject classifier $\Omega$.
    \end{enumerate}
\end{definition}

Topos theory generalizes set theory and provides a framework for mathematical logic and geometry. A topos can be seen as a universe of varying contexts, each equipped with its own internal logic.

\subsubsection{Internal Logic of a Topos}

Each topos $\mathcal{E}$ has an internal language that is intuitionistic, meaning that it does not necessarily satisfy the law of excluded middle (LEM). The internal logic allows for reasoning about objects and morphisms within the topos as if they were sets and functions, but with a more flexible logical foundation.

\subsubsection{Sheaves and Presheaves}

\begin{definition}
    A \textbf{presheaf} on a category $\mathcal{C}$ is a functor $F \colon \mathcal{C}^{\text{op}} \to \mathbf{Set}$.
\end{definition}

\begin{definition}
    A \textbf{sheaf} on a topological space $X$ is a presheaf that satisfies the sheaf axioms, which ensure that local data can be uniquely glued together to form global data.
\end{definition}

Sheaf theory provides a way to systematically track locally defined data attached to the open sets of a topological space, enabling the formalization of context-dependent truth.

\subsection{Adjunctions in Category Theory}

\begin{definition}
    An \textbf{adjunction} between two categories $\mathcal{C}$ and $\mathcal{D}$ consists of two functors:
    \[
        \mathcal{A} \colon \mathcal{C} \to \mathcal{D} \quad \text{and} \quad \mathcal{P} \colon \mathcal{D} \to \mathcal{C}
    \]
    and a natural bijection
    \[
        \phi_{CB} \colon \text{Hom}_{\mathcal{D}} ( \mathcal{A}(C), B ) \cong \text{Hom}_{\mathcal{C}} (C, \mathcal{P}(B))
    \]
    for all objects $C$ in $\mathcal{C}$ and $B$ in $\mathcal{D}$.
\end{definition}

In this context, $\mathcal{A}$ is the \textbf{left adjoint} and $\mathcal{P}$ is the \textbf{right adjoint} and we write $\mathcal{A} \dashv \mathcal{P}$.

\section{Cognition as adjunction}

\subsection{Defining the categories}

\begin{definition} The category $\mathbf{Con}$ (Contexts):

    \begin{itemize}
        \item \textbf{Objects}: Contexts or situations in which cognition occurs.
        \item \textbf{Arrows}: Transitions or relationships between contexts.
    \end{itemize}
\end{definition}

\begin{example} A context could be a visual scene, and a morphism could represent the act of shifting attention within that scene. \end{example}

\begin{definition} The category $\mathbf{Beh}$ (Behaviors):

    \begin{itemize}
        \item \textbf{Objects}: Behaviors or actions taken by an agent.
        \item \textbf{Arrows}: Processes transforming one behavior into another.
    \end{itemize}
\end{definition}

\begin{example} A behavior could be reaching for an object, and a morphism could represent the modification of that action due to environmental feedback. \end{example}

\subsection{Modeling action and perception as functors}
We model action and perception as functors between these categories:

\begin{itemize}
    \item \textbf{Action Functor} $\mathcal{A} \colon \mathbf{Con} \to \mathbf{Beh}$: Maps contexts to behaviors.
    \item \textbf{Perception Functor} $\mathcal{P} \colon \mathbf{Beh} \to \mathbf{Con}$: Maps behaviors to resulting contexts.
\end{itemize}

\subsection{Establishing the Adjunction}

The adjunction $\mathcal{A} \dashv \mathcal{P}$ captures the reciprocal relationship between action and perception.

\begin{theorem}
    For any context $C$ and behavior $B$ there is a bijection
    \[
        \phi_{C, B} \colon \text{Hom}_{\textbf{Beh}} ( \mathcal{A} (C), B) \cong Hom_{\textbf{Con}} (C, \mathcal{P}(B))
    \]
    between such behavioral and contextual transformations that is natural in both $C$ and $B$.
\end{theorem}
\begin{proof}
    This is straight up the definition of $\mathcal{A} \dashv \mathcal{P}$.
\end{proof}

This adjunction formalizes the idea that actions are generated from contexts and, in turn, behaviors influence subsequent contexts through perception. It embodies the enactivist view that cognition is a continuous loop of action and perception.

In this diagram, the dotted arrow represents the process of enaction, combining action and perception into a cohesive cognitive process.

\section{The Empty Self and \emph{Śūnyatā}}

\subsection{Initial Object in Category Theory}

\begin{definition} An \textbf{initial object} in a category $\mathcal{C}$ is an object $0$ such that for every object $X$ in $\mathcal{C}$, there exists a unique morphism $0 \to X$. \end{definition}

\begin{example} In $\mathbf{Set}$, the empty set $\emptyset$ is the initial object. To see this, note that \end{example}

\subsection{Self as the Initial Object}

We propose that the self can be modeled as the initial object in $\mathbf{Con}$.

\begin{remark} The self, being the origin of all experiences, maps uniquely to every context but lacks inherent structure within this framework, aligning with the notion of an initial object. \end{remark}

\subsection{Connection to \emph{Śūnyatā}}

In Buddhist philosophy, \emph{śūnyatā} refers to the emptiness of inherent existence in all phenomena, including the self \cite{garfield1995}.

\begin{quote} "Form is emptiness, emptiness is form." -- \textit{Heart Sutra} \end{quote}

Our model reflects this by representing the self as devoid of intrinsic properties, defined only through its relations (morphisms) to contexts.

\subsection{Philosophical Implications}

This conceptualization challenges the traditional notion of a fixed, representational self and supports a relational understanding of identity.

\section{Examples and Applications}

\subsection{A Simple Cognitive Process}

Consider an agent navigating a maze.

\begin{itemize} \item \textbf{Contexts}: The agent's current location and perception of the maze. \item \textbf{Behaviors}: Movements such as turning left or right. \item \textbf{Action Functor $\mathcal{A}$}: Given a context (current location), the functor maps to a behavior (choose direction). \item \textbf{Perception Functor $\mathcal{P}$}: Given a behavior (movement), the functor maps to a new context (new location). \end{itemize}

The adjunction captures how the agent's decision to move influences its perception and how that perception informs future actions.

\subsection{Expanded Example: Language Acquisition}

Traditional models of language acquisition often rely on a representational framework where the brain is seen as storing internal models of grammar and vocabulary, which are then retrieved and applied during language use. In contrast, our adjunction-based model emphasizes the dynamic interaction between the agent and its linguistic environment.

For instance, in language learning:

\begin{itemize}
    \item \textbf{Contexts}: Linguistic environments (e.g., hearing spoken language, reading text).
    \item \textbf{Behaviors}: Speech acts (e.g., speaking, writing, gesturing).
    \item \textbf{Action Functor} $\mathcal{A}$: Maps contexts (hearing a word) to behaviors (attempting to use the word).
    \item \textbf{Perception Functor} $\mathcal{P}$: Maps behaviors (the act of speaking) to resulting contexts (the listener’s reaction, or self-correction based on feedback).
\end{itemize}

In our model, this interaction forms a natural adjunction: for every linguistic context, there is a corresponding set of actions that influence the learner's perception, which in turn modifies future actions. Unlike rigid representational models, this adjunction-based framework captures the dynamic, relational nature of language acquisition as proposed by enactivism.

\subsection{Implications for Artificial Intelligence}

Modeling cognition in this way can inform the development of AI systems that learn and adapt through interaction, rather than relying solely on pre-programmed representations.

\section{Implications for Cognitive Science and Philosophy}

\subsection{Hard problem of consciousness}
This framework not only aligns with enactivism but also contributes to ongoing debates in philosophy of mind, particularly concerning the ``hard problem'' of consciousness (Chalmers). By rejecting the notion of a fixed, representational self and adopting a relational view through the adjunction framework, we propose that consciousness emerges through the ongoing interaction between the self and its contexts.

This relational view contrasts with modern theories such as Dennett's "self as a narrative" and Metzinger's "Ego Tunnel," which treat the self as either an illusion or a construct. Instead, our model suggests that the self, as the initial object, is devoid of intrinsic structure and gains meaning only through its dynamic relations to contexts --- paralleling both enactivism and Eastern philosophies such as \emph{śūnyatā}.


\subsection{Context-Dependence and Situatedness}
By modeling contexts explicitly, we highlight the importance of situatedness in cognitive processes, aligning with embodied cognition theories \cite{clark1997}.

\subsection{Bridging Eastern and Western Thought}

The correspondence between the empty self and \emph{śūnyatā} offers a philosophical bridge that enriches both traditions.

\section{Related Work}

\subsection{Enactivism in Cognitive Science}

Our work builds upon the foundational ideas of enactivism \cite{varela1991}, extending them with mathematical rigor.

\subsection{Category Theory in Cognitive Models}

Previous attempts to apply category theory to cognition include conceptual spaces \cite{gardenfors2004} and neural network architectures \cite{spivak2014}.

\subsection{Philosophical Perspectives on the Self}

The concept of the self as relational is explored in both Western \cite{heidegger1962} and Eastern philosophies \cite{garfield1995}.

\section{Future Work}

\subsection{Formalizing the Categories}

Further research will focus on precisely defining the objects and morphisms in $\mathbf{Con}$ and $\mathbf{Beh}$, potentially using empirical data.

\subsection{Potential Empirical Validation}
While this framework is currently theoretical, it lends itself to empirical validation in domains such as cognitive neuroscience or AI. For instance, one can design experiments where an agent (biological or artificial) interacts with an environment, and one can track how action-perception cycles form adjunctions that influence behavior over time.

Specifically, reinforcement learning systems, where an agent learns through interaction with an environment, could be adapted to model the adjunction $\mathcal{A} \dashv \mathcal{P}$. By comparing this adjunction-based model to traditional AI systems that rely on internal representations, we can empirically investigate whether the adjunction model provides a more accurate or efficient description of cognitive processes.

\subsection{Extensions to Other Cognitive Domains}

Exploring how this framework applies to social cognition, emotions, and consciousness.

\section{Conclusion}

This work has presented a novel framework to expresses cognition as an adjunction between action and perception, capturing the dynamic interplay central to enactivism. By conceptualizing the self as the initial object, we align with the philosophical concept of \emph{śūnyatā}, offering a relational understanding of identity. This interdisciplinary approach provides new insights into cognitive science and opens avenues for future research.

\section*{Acknowledgments}

% Optional: Acknowledge individuals or institutions that contributed to the work.
The author kindly thanks the University of Chicago's mathematics REU program --- and, in particular, Peter May --- for cultivating a wonderful environment for learning, in addition to Michael Smith, Sarah Schienman, Milo Korman, Jay Cullen, Jakob Lövhall, David ``Buddy'' Ryan, Wyatt Hope, Charles Hourihan and E. Glen Weyl for their sage counsel.

% Example references. Replace with your actual references.
\begin{thebibliography}{9}

    \bibitem{varela1991}
    Varela, F. J., Thompson, E., \& Rosch, E. (1991). \textit{The Embodied Mind: Cognitive Science and Human Experience}. MIT Press.

    \bibitem{esterson1985}
    Esterson, A. (1985). \textit{The Affirmation of Experience: A contribution towards a science of social situations}. https://www.szasz.com/estersone.pdf

    \bibitem{garfield1995}
    Garfield, J. L. (1995). \textit{The Fundamental Wisdom of the Middle Way: Nāgārjuna's Mūlamadhyamakakārikā}. Oxford University Press.

    \bibitem{clark1997} Clark, A. (1997). \textit{Being There: Putting Brain, Body, and World Together Again}. MIT Press.

    \bibitem{fodor1980} Fodor, J. A. (1980). \textit{Methodological Solipsism Considered as a Research Strategy in Cognitive Psychology}. Behavioral and Brain Sciences, 3(1), 63-73.

    \bibitem{thompson2007} Thompson, E. (2007). \textit{Mind in Life: Biology, Phenomenology, and the Sciences of Mind}. Harvard University Press.

    \bibitem{pylyshyn1987} Pylyshyn, Z. W. (1987). \textit{The Robot's Dilemma: The Frame Problem in Artificial Intelligence}. Ablex Publishing.

    \bibitem{thompson2007} Thompson, E. (2007). \textit{Mind in Life: Biology, Phenomenology, and the Sciences of Mind}. Harvard University Press.

    \bibitem{wittgenstein1953} Wittgenstein, L. (1953). \textit{Philosophical Investigations}. Blackwell Publishing.

    \bibitem{heidegger1962} Heidegger, M. (1962). \textit{Being and Time}. (J. Macquarrie, E. Robinson, Trans.). Harper \& Row.

    \bibitem{spivak2014}
    Spivak, D. I. (2014). \textit{Category Theory for the Sciences}. MIT Press.

    \bibitem{gardenfors2004}
    Gärdenfors, P. (2004). \textit{Conceptual Spaces: The Geometry of Thought}. MIT Press.

    \bibitem{mahany2012}
    Mahany, C. L. (2012). \textit{Elementary Topos Theory and Intuitionistic Logic}. REU, University of Chicago. https://math.uchicago.edu/~may/REU2012/REUPapers/Mahany.pdf

    \bibitem{west2012}
    West, R. F., Meserve, R. J., \& Stanovich, K. E. (2012). Cognitive sophistication does not attenuate the bias blind spot. \textit{Journal of Personality and Social Psychology}, 103(3), 506-519. https://doi.org/10.1037/a0028857

\end{thebibliography}

\end{document}